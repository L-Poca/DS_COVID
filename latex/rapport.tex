\documentclass[11pt, a4paper]{article}

% --- PACKAGES ESSENTIELS ---
\usepackage[utf8]{inputenc}       % Gérer les accents
\usepackage[T1]{fontenc}          % Encodage de la police
\usepackage[french]{babel}        % Règles typographiques françaises
\usepackage{graphicx}             % Inclure des images
\usepackage{amsmath}              % Pour les maths
\usepackage{geometry}             % Régler les marges
\geometry{a4paper, margin=2.5cm}

% --- PACKAGES POUR LA MISE EN FORME ---
\usepackage{booktabs}             % Pour de jolis tableaux
\usepackage{longtable}            % Pour les tableaux sur plusieurs pages
\usepackage{parskip}              % Aère le texte en sautant une ligne entre les paragraphes

% --- GESTION DES LIENS ET RÉFÉRENCES ---
\usepackage{hyperref}             % Pour les liens cliquables (PDF)
\hypersetup{
    colorlinks=true,
    linkcolor=blue,
    filecolor=magenta,      
    urlcolor=cyan,
    pdftitle={Analyse de Radiographies pulmonaires Covid-19},
    pdfpagemode=FullScreen,
}
\usepackage[all]{hypcap} % Fait pointer les liens vers le début de la figure, pas la légende

% --- CONFIGURATION DES CHEMINS ---
% Indique à LaTeX où chercher les images, évite de répéter le chemin
\graphicspath{{media/}}

% ==============================================================================
% --- DÉBUT DU DOCUMENT ---
% ==============================================================================
\begin{document}

% --- PAGE DE TITRE ---
\begin{titlepage}
    \centering
    
    % Logo de l'école/université (ajustez la taille si besoin)
    \includegraphics[width=0.8\textwidth]{image1.png}
    
    \vspace{2cm} % Espace vertical
    
    % Titre du projet
    {\Huge \bfseries Analyse de Radiographies pulmonaires Covid-19 \par}
    
    \vspace{1.5cm} % Espace vertical
    
    % Image de couverture
    \includegraphics[width=0.6\textwidth]{image2.png}
    
    \vspace{2cm} % Espace vertical
    
    % Auteurs et encadrants
    \begin{minipage}{0.4\textwidth}
        \begin{flushleft} \large
            \emph{Présenté par :}\\
            \vspace{0.5cm}
            Cirine Bouamrane \\
            Léna Bacot \\
            Steven Moire \\
            Rafael Cepa
        \end{flushleft}
    \end{minipage}
    \begin{minipage}{0.4\textwidth}
        \begin{flushright} \large
            \emph{Encadré par :}\\
            \vspace{0.5cm}
            Nicolas Mormiche
        \end{flushright}
    \end{minipage}

    \vfill % Pousse le contenu suivant en bas de la page
    
    % Date
    {\large \today \par}
    
\end{titlepage}


\newpage
\tableofcontents % Génère la table des matières automatiquement
\newpage

% --- CORPS DU RAPPORT ---
\section{Introduction}
\label{sec:introduction}

Le COVID-19 (maladie à coronavirus 2019) est une maladie infectieuse causée par le virus SARS-CoV-2. Apparue pour la première fois en décembre 2019 dans la ville de Wuhan, en Chine, elle s'est rapidement propagée à travers le monde, provoquant une pandémie mondiale sans précédent.

Les symptômes du COVID-19 varient considérablement, allant de légers symptômes grippaux (fièvre, toux, fatigue) à des formes graves caractérisées par des difficultés respiratoires aiguës, une pneumonie sévère, et dans certains cas, la mort. Les personnes âgées ou celles souffrant de comorbidités (maladies cardiovasculaires, diabète, maladies respiratoires chroniques) sont particulièrement vulnérables aux formes graves de la maladie.

Face à la propagation rapide du virus, le diagnostic précoce et précis est devenu un enjeu majeur pour la gestion de la crise sanitaire. Les tests RT-PCR, bien que considérés comme la référence, présentent des limites en termes de disponibilité, de coût et de délais d'obtention des résultats.

Dans ce contexte, l'imagerie médicale, et en particulier la radiographie pulmonaire (CXR), est apparue comme un outil complémentaire précieux. Moins coûteuse et plus accessible que le scanner (CT-scan), elle peut révéler des signes caractéristiques de la pneumonie virale induite par le COVID-19, tels que des opacités en verre dépoli.

Cependant, l'interprétation des radiographies pulmonaires dépend fortement de l'expertise du radiologue et peut être subjective. L'intelligence artificielle, et plus spécifiquement les modèles de \emph{Deep Learning} comme les réseaux de neurones convolutifs (CNN), offre une opportunité unique d'automatiser et de fiabiliser cette analyse. Ces modèles sont capables d'apprendre à partir de milliers d'images pour identifier des motifs complexes souvent invisibles à l'œil humain.

L'objectif de ce projet est donc de développer un modèle de classification d'images basé sur le \emph{Deep Learning}, capable de distinguer avec une grande précision les radiographies pulmonaires de patients atteints du COVID-19, de celles de patients sains ou atteints d'une pneumonie virale non-COVID. Un tel outil pourrait aider à accélérer le diagnostic, à désengorger les services de radiologie et à améliorer la prise en charge des patients dans les zones à ressources limitées.

% ... on ajoutera les autres sections ici au fur et à mesure


\end{document}
